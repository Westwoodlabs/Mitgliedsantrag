\documentclass[ngerman,a4paper]{article}
\usepackage[T1]{fontenc}
\usepackage{libertine}
\usepackage{babel}
\usepackage[utf8]{inputenc}
\usepackage{graphicx}
\usepackage{multicol}
\usepackage[%
    driver=pdftex,
    nopro,
    web={pro,
        tight,
        german,
        nodirectory,
        usesf
        },
    eforms,
    aebxmp,
    attachsource={tex}
]{aeb_pro}
\usepackage{calc}
\usepackage{eforms}
\usepackage{array}
\usepackage{makecell}
\usepackage{scrextend}

%\screensize{297mm}{210mm} % height, width
\margins{.6in}{.6in}{.5in}{.5in}% left,right,top, bottom

% References
% http://mirror.las.iastate.edu/tex-archive/macros/latex/contrib/aeb_pro/aeb_pro/doc/aebpro_man.pdf
% https://ctan.net/macros/latex/contrib/acrotex/doc/eformman.pdf
% https://www.adobe.com/content/dam/acom/en/devnet/acrobat/pdfs/FormsAPIReference.pdf
% https://www.binary-kitchen.de/wiki/lib/exe/fetch.php?media=beitrittserklaerung_20190403.pdf
% https://www.dante.de/wp-content/uploads/2019/06/AntragStandardSepa.pdf


%% Layoutparameter
\parindent 0pt
\parskip 3pt
\pagestyle{empty}

\newcommand{\infoInput}[2][8cm]{\stepcounter{infoLineNum}%
		\makebox[((#1)+0.1in)][l]{\makebox[0pt][l]{\kern4pt\raisebox{.75ex}{\textField[\W0\BC{}\BG{}\TU{#2}]{name\theinfoLineNum}{#1}{12bp}}}\dotfill}}

\newcommand{\infoInputNum}[2][3cm]{\stepcounter{infoLineNum}%
	\makebox[((#1)+0.1in)][l]{\makebox[0pt][l]{\kern4pt\raisebox{.75ex}{\textField[\W0\BC{}\BG{}\TU{#2}\AA{%
				\AAKeystroke{AFNumber_Keystroke(0,0,0,0,"",true)}
				\AAFormat{AFNumber_Format(0,0,0,0,"",true)}}
			]{name\theinfoLineNum}{#1}{12bp}}}\dotfill}}
		
\newcommand{\infoInputDate}[2][3cm]{\stepcounter{infoLineNum}%
	\makebox[((#1)+0.1in)][l]{\makebox[0pt][l]{\kern4pt\raisebox{.75ex}{\textField[\W0\BC{}\BG{}\TU{#2}\AA{%
				\AAKeystroke{AFDate_KeystrokeEx("yy-mm-dd")}
				\AAFormat{AFDate_Format("yy-mm-dd")}}
			]{name\theinfoLineNum}{#1}{12bp}}}\dotfill}}


\newcommand{\signarea}{
	\begin{minipage}{0.35\textwidth}
		\begin{center}
			\infoInput[\textwidth]{Ort}
				
			Ort
		\end{center}
	\end{minipage}\hfill
	\begin{minipage}{0.15\textwidth}
		\begin{center}
			\infoInputDate[\textwidth]{Datum}
			
			Datum
		\end{center}
	\end{minipage}\hfill
	\begin{minipage}{0.35\textwidth}
		\begin{center}
			\infoInput[\textwidth]{Unterschrift}
			
			Unterschrift
		\end{center}
	\end{minipage}
}


\DeclareDocInfo{
	title=Antrag auf Mitgliedschaft - Westwoodlabs e.V.,
	author=Westwoodlabs e.V.
}

\begin{document}

\centerline{\includegraphics[scale=0.5]{logo}}

\begin{center}
	\LARGE{Antrag auf Mitgliedschaft im Verein Westwoodlabs e.V.}
\end{center}

\newcounter{infoLineNum}
\setcounter{infoLineNum}{0}

\vspace{5pt}
\textbf{Stammdaten}\\
\begin{tabular}{lp{5.5in}}
    Nachname, Vorname:   				& \infoInput{Nachname, Vorname}\\[6pt]
    Geburtsdatum:   					& \infoInput{Geburtsdatum}\\[6pt]
    Straße und Hausnummer:              & \infoInput{Strasse Hausnummer}\\[6pt]
    PLZ und Ort:                        & \infoInput{PLZ Ort}\\[6pt]
    E-Mail:                             & \infoInput{E-Mail}\\[6pt]
\end{tabular}

\newcounter{checkboxcounter}
\setcounter{checkboxcounter}{0}
\def\firstCk{\stepcounter{checkboxcounter}\checkBox{checkbox\thecheckboxcounter}{11bp}{11bp}{ja}}
\renewcommand{\labelitemi}{\firstCk}

\vspace{10pt}

\textbf{Mitgliedschaft}
\begin{itemize}
	\item Vollmitglied\footnote{Vollmitglieder zahlen einen Mindestbeitrag von 20,-€ pro Monat. Der Beitrag kann auf freiwilliger Basis erhöht werden.}
	\item Vollmitglied ermäßigt\footnote{Ermäßigte Vollmitglieder zahlen einen Mindestbeitrag von 10,-€ pro Monat. Eine solche Ermäßigung gibt es für Studierende, Rentner/innen, Arbeitslose, Zivil- und Wehrdienstleistende. Bitte eine entsprechende Bescheinigung mitsenden, da ansonsten eine nachträgliche Hochstufung in die Gruppe \textit{Vollmitgliedschaft} erfolgt. Der Beitrag kann auf freiwilliger Basis erhöht werden.} (bitte Bescheinigung beilegen)
	\item Fördermitgliedschaft\footnote{Fördermitglieder zahlen einen Mindestbeitrag von 5,-€ pro Monat. Fördermitglieder erhalten keinerlei Stimmrecht. Der Beitrag kann auf freiwilliger Basis erhöht werden.}
\end{itemize}

\vspace{5pt}

\textbf{Mitgliedsbeitrag\footnote{Bitte beachte, wenn du mit deinem Mitgliedsbeiträgen um mehr als sechs Monate im Rückstand bist, kann dies zur Kündigung und zum Ausschluss aus dem Verein führen.}}
\begin{addmargin}{1em}
	\infoInputNum[0.5in]{Mehrbeitrag}€ pro Monat\
\end{addmargin}



\vspace{5pt}

\textbf{Zahlungsmodalitäten}
\begin{itemize}
\item Lastschrift (Bitte unterschriebene Einzugsermächtigung beilegen, siehe zweite Seite)
\item Überweisung oder Dauerauftrag (Bitte nur in Ausnahmefällen. Fällig zum 01. jeden Monats)
\end{itemize}

\vspace{20pt}

Ich beantrage hiermit die Mitgliedschaft im Verein Westwoodlabs e.V. und erkenne die Satzung und Ziele des Vereins mit meiner Unterschrift an. Über die Aufnahme in Verein entscheidet der Vorstand. Eine Rückmeldung erhältst du zeitnah per E-Mail. Sollte dein Antrag angenommen werden, bekommst du eine Willkommens-E-Mail.

\vspace{40pt}

\signarea

\vspace{40pt}

Sende den vollständig ausgefüllten Antrag bitte an:
\begin{center}
	\begin{tabular}{c>{\centering\arraybackslash}m{3cm}c}
		\makecell{vorstand@westwoodlabs.de \\ (eingescannt oder mit \\ Qualifizierter elektronischer Signatur)} & oder & \makecell{Westwoodlabs e.V. \\ c/o Patric Steffen \\ Friedrichstraße 13 \\ 57639 Rodenbach-Udert}\\
	\end{tabular}
\end{center}

\vspace{10pt}


%% Rückseite
\newpage

\renewcommand\thesubsection{\arabic{subsection}}

%\begin{flushright}
Westwoodlabs e.V.\\
c/o Patric Steffen\\
Friedrichstraße 13\\
57639 Rodenbach-Udert\\
%\end{flushright}

Westerwald Bank eG Volks- und Raiffeisenbank\\
IBAN:\\
BIC: GENODE51WW1\\
Gläubiger-Identifikationsnummer: DE76WWL00002322139\\[5pt]
\\[1cm]

\textbf{\textsf{V\,E\,R\,T\,R\,A\,G}}\\

\bigskip
\textbf{Mitglied}\\
\\
\begin{tabular}{p{4cm}l}
	Kontoinhaber:   					& \infoInput{Kontoinhaber}\\[6pt]
	Straße und Hausnummer:              & \infoInput{Strasse Hausnummer}\\[6pt]
	PLZ und Ort:                        & \infoInput{PLZ Ort}\\[6pt]
\end{tabular}

\bigskip
\textbf{Kreditinstitut}\\
\\
\begin{tabular}{p{4cm}l}
	Name d. Kreditinstitutes:		& \infoInput{Kreditinstitut}\\[6pt]
	IBAN:       					& \infoInput{IBAN}\\[6pt]
	BIC:       						& \infoInput{BIC}\\[6pt]
\end{tabular}

\bigskip
\textbf{Erteilung einer Einzugsermächtigung und eines SEPA-Lastschriftmandats}

\bigskip
\textbf{1. Einzugsermächtigung}

Ich ermächtige Westwoodlabs e.V. widerruflich, die von mir zu entrichtenden Zahlungen bei fälligkeit 
durch Lastschrift von meinem Konto einzuziehen.

\bigskip
\textbf{2. SEPA-Lastschriftmandat}

Ich ermächtige Westwoodlabs e.V., Zahlungen von meinem Konto mittels Lastschrift einzuziehen. 
Zugleich weise ich mein Kreditinstitut an, die von Westwoodlabs e.V. auf mein Konto 
gezogenen Lastschriften einzulösen.


\emph{Hinweis:} Ich kann innerhalb von acht Wochen, beginnend mit dem Belastungsdatum,
die Erstattung des belasteten Betrages verlangen. Es gelten dabei die mit meinem
Kreditinstitut vereinbarten Bedingungen.

\bigskip
Vor dem ersten Einzug einer SEPA-Basislastschrift wird mich Westwoodlabs e.V über den Einzug in dieser Verfahrensart unterrichten und mir die Mandatsreferenz mitteilen.\\

\vspace{50pt}

\signarea
\end{document} 
